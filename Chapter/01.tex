\documentclass[../Koblitz.tex]{subfiles}

\begin{document}

\chapter{\texorpdfstring{$p$}{p}-adic numbers}

\section*{\S2 (pp. 6-8)}

\subsection*{Exercise 1.1}

\subsection*{Exercise 1.2}

Observe that $\|1\|=\|1\cdot1\|=\|1\|^2$ and so $\|1\|=1$. On the other hand, $1=\|1\|=\|-1\|^2$ and so $\|-1\|=1$.

Suppose $\dnr$ is non-Archimedean. We claim that $\|x_1+\cdots+x_k\| \leq \max{\{\|x_1\|,\ldots,\|x_k\|\}}$. Induction on $k$. When $k=3$, we know $\|x_1+x_2+x_3\| \leq \max\{\|x_1+x_2\|,\|x_3\|\}$. In the former case, we have $\|x_1+x_2+x_3\| \leq \|x_1+x_2\| \leq \max\{\|x_1\|,\|x_2\|\} \leq \max\{\|x_1\|,\|x_2\|,\|x_3\|\}$. And in the latter case, we have $\|x_1+x_2+x_3\| \leq \|x_3\| \leq \max\{\|x_1\|,\|x_2\|,\|x_3\|\}$.

Consider $\|x_1+\cdots+x_k\| \leq \max\{\|x_1+\cdots+x_{k-1}\|,\|x_k\|\}$. In the former case, we have $\|x_1+\cdots+x_k\| \leq \|x_1+\cdots+x_{k-1}\| \leq \max\{\|x_1\|,\ldots,\|x_{k-1}\|\} \leq \max\{\|x_1\|,\ldots,\|x_{k-1}\|,\|x_k\|\}$. (We've used the induction hypothesis in the second inequality.) And in the latter case, we have $\|x_1+\cdots+x_k\| \leq \|x_k\| \leq \max\{\|x_1\|,\ldots,\|x_k\|\}$.

Now, $\|n\|=\|1+\cdots+1\| \leq \max\{\|1\|,\ldots,\|1\|\}=1$. So we are done.

\subsection*{Exercise 1.3}

Given $x,y\in F$. By the binomial theorem and the assumption, we have 
\begin{align*}
\|x+y\|^n &= \|(x+y)^n\| = \left\|\sum_{k=0}^n \binom{n}{k}x^{n-k}y^k\right\| \leq \sum_{k=0}^n \left\|\binom{n}{k}\right\|\|x^{n-k}\|\|y^k\| \\
&\leq \sum_{k=0}^n \|x\|^{n-k} \|y\|^k \leq (n+1)\max\{\|x\|,\|y\|\}^n
\end{align*}
The result now follows by taking $n$-th root both sides and letting $n\to\infty$.

\subsection*{Exercise 1.4}

$(\Leftarrow)$ Suppose there exists $x\in F$ s.t. $\|x\| < 1$ and $\|x-1\| < 1$. Then we have $1=\|1\|=\|1-x+x\|\leq \max\{\|1-x\|,\|x\|\} < 1$, a contradiction.

$(\Rightarrow)$ Suppose $\dnr$ is Archimedean, then by definition, $\exists x,y\in F$ s.t. $\|x+y\|>\max\{ \|x\|,\|y\|\}$. Set $a:=x/(x+y)$, then $\|a\|=\|x/(x+y)\| < \|x/\max\{ \|x\|,\|y\|\} \| \leq 1$. And similarly, $\|a-1\|=\|y/(x+y)\| < \|y/\max\{ \|x\|,\|y\|\} \| \leq 1$, a contradiction.

\subsection*{Exercise 1.5}

\subsection*{Exercise 1.6}

Assume $0<\rho<1$. We check the given function is non-Archimedean, i.e., $\rho^{\ord_p(x+y)}\leq \max\{\rho^{\ord_p(x)},\rho^{\ord_p(y)}\}$. Given $x,y\in\QQ$, we may assume $x,y,x+y\neq0$ because otherwise this holds trivially. From the proof of the Prop in page 2, we know $\ord_p(x+y) \geq \min\{\ord_p(x),\ord_p(y)\}$ and so $-\ord_p(x+y) \leq \max\{-\ord_p(x),-\ord_p(y)\}$. Write $\rho=1/P$, then $$\rho^{\ord_p(x+y)} = P^{-\ord_p(x+y)} \leq \max\{P^{-\ord_p(x)},P^{-\ord_p(y)}\} = \max\{\rho^{\ord_p(x)},\rho^{\ord_p(y)}\}$$

For $\rho=1$, we have $\rho^{\ord_p(x)}=1$ for all $x\neq0$. So the given function is the trivial norm.

Now suppose $\rho>1$, we claim that there exist $x,y\in\QQ$ s.t. $\rho^{\ord_p(x+y)} > \rho^{\ord_p(x)}+\rho^{\ord_p(y)}$, i.e., the triangle inequality fails. And hence the given function is not a norm.

Take $k\in\NN$ large enough s.t. $\sqrt[k]{2}<\rho$. So we have $\rho^k>2$. Now, let $x=1$ and $y=p^k-1$. Then $\ord_p(x)=\ord_p(y)=0$ and $\ord_p(x+y)=k$. Hence, we have $\rho^{\ord_p(x+y)} = \rho^k >2 = \rho^0 + \rho^0 = \rho^{\ord_p(x)}+\rho^{\ord_p(y)}$.

\begin{comment}
Take $k\in\NN$ large enough s.t. $\sqrt[k]{2}<\rho$. Note that we can take $k$ is even. So we have $\rho^{k+1}>2\rho$. Now let $x=p\cdot(p^{k/2}+1)^{-1}$ and $y=p\cdot(p^{k/2}-1)$. Then $x,y\in\QQ$, and it's easy to check that $\ord_p(x)=\ord_p(y)=1$ and $\ord_p(x+y)=k+1$. So we have $$\rho^{\ord_p(x+y)} = \rho^{k+1} >2\rho = \rho+\rho = \rho^{\ord_p(x)}+\rho^{\ord_p(y)}$$
\end{comment}

\subsection*{Exercise 1.7}

WLOG, assume $p_1<p_2$. Suppose on the contrary, $\snr_{p_1}$ is equivalent to $\snr_{p_2}$. Consider the sequence $\{p_1^n\}_{n=1}^\infty$. We claim that this is Cauchy w.r.t. $\snr_{p_1}$. Given $\epsilon>0$, take $n_\epsilon\in\NN$ s.t. $1/p_1^{n_\epsilon}<\epsilon$. If $n>m\geq n_\epsilon$, then $|p_1^n-p_1^m|_{p_1} = |p_1^m(p_1^{n-m}-1)|_{p_1} = 1/p_1^m \leq 1/p_1^{n_\epsilon} < \epsilon$.

Since $\snr_{p_1}$ is equivalent to $\snr_{p_2}$, we know $\{p_1^n\}_{n=1}^\infty$ is Cauchy w.r.t. $\snr_{p_2}$. So given $\epsilon=1$, $\exists N\in\NN$ s.t. $n>m\geq N$, we have $|p_1^n-p_1^m|_{p_2}<1$. Take $n=N+1,m=N$ and set $x:=p_1^{N+1}-p_1^N$, then $0<|x|_{p_2} = (1/p_2)^{\ord_{p_2}(x)} < 1$. This implies $\ord_{p_2}(x) \geq1$ and so $p_2\mid x$. From this we have $p_2\mid p_1^N$ or $p_2\mid p_1-1$. Note that the latter case is impossible because we assume $p_1<p_2$, so we have $p_2\mid p_1^N$ and hence $p_2=p_1$, a contradiction.

\subsection*{Exercise 1.8}

\subsection*{Exercise 1.9}

\subsection*{Exercise 1.10}

\subsection*{Exercise 1.11}

\subsection*{Exercise 1.12}

Induction on $N$. When $N=1$, we have $\ord_p(p!)=\sum_{k=1}^p \ord_p(k)$. Since $\ord_p(k)=0$ for $k=1,\ldots,p-1$, and $\ord_p(p)=1$. So $\ord_p(p!)=1$.

Now, suppose $\ord_p((p^N)!)=1+p+\cdots+p^{N-1}$. We need to show that $\ord_p((p^{N+1})!) = 1+p+\cdots+p^N$. Note that $$\ord_p((p^{N+1})!) = \sum_{k=1}^{p^N} \ord_p(k) + \sum_{k=p^N+1}^{p^{N+1}} \ord_p (k) = 1+p+\cdots+p^{N-1} + \sum_{k=p^N+1}^{p^{N+1}} \ord_p (k)$$ So it's sufficient to show $\sum_{k=p^N+1}^{p^{N+1}} \ord_p (k) = p^N$.

The integers $k$ between $p^N+1$ and $p^{N+1}$ can be written as $p^N+m$ where $m=1,2,\ldots,p^N(p-1)$. And the corresponding order is non-zero iff $k=p^N+t\cdot p$ where $t=1,2,\ldots,p^{N-1}(p-1)$. Let $\Tcal_n :=\{ t \mid \ord_p(p^N+tp) \geq n\} \subset \{1,2,\ldots,p^{N-1}(p-1)\}$. Then it's easy to see that
\begin{align*}
&\Tcal_1 = \{1,2,3,\ldots,p^{N-1}(p-1)\} \\
&\Tcal_2 = \{p,2p,3p,\ldots,p^{N-2}(p-1)p\} \\
&\Tcal_3 = \{p^2,2p^2,3p^2,\ldots,p^{N-3}(p-1)p^2\} \\
&\quad\; \: \vdots \\
&\Tcal_N = \{p^{N-1},\ldots,(p-1)p^{N-1}\} \\
&\Tcal_{N+1} = \{(p-1)p^{N-1}\}
\end{align*}
The key observation is that for each $t$, $\ord_p(p^N+tp) = \#\{n\mid t\in\Tcal_n\}$. Because if we write $t=p^ks$ where $p\nmid s$, then $\ord_p(p^N+tp)=\ord_p(p^N+p^{k+1}s)=k+1$. And $t\in\Tcal_1,\ldots,\Tcal_{k+1}$ as $t=p^ks=p^{k-1}s\cdot p=p^{k-2}s\cdot p^2=\cdots=s\cdot p^k$, respectively.

From this observation, we have 
\begin{align*}
\sum_{k=p^N+1}^{p^{N+1}} \ord_p (k) &= \sum_{t=1}^{p^{N-1}(p-1)} \ord_p(p^N+tp) = \sum_{n=1}^{N+1} \#(\Tcal_n) \\
&= p^{N-1}(p-1) + p^{N-2}(p-1) + \cdots + (p-1) + 1 = p^N
\end{align*}

\subsection*{Exercise 1.13}

Note that $a=0$ is trivial, and $a=1$ has been done in Ex 1.12. So fix $2\leq a\leq p-1$, we claim that for any $1\leq k\leq p^N$ and $1\leq\alpha\leq a-1$, we have $\ord_p(k)=\ord_p(\alpha p^N+k)$. This will imply the result immediately.

Write $k=p^ts$ where $0\leq t=\ord_p(k)\leq N$ and $p\nmid s$. Then
\begin{align*}
\ord_p(\alpha p^N+k) &= \ord_p(\alpha p^N+p^ts) = \ord_p(p^t(\alpha p^{N-t}+s) \\
&= \ord_p(p^t) + \ord_p(\alpha p^{N-t}+s) = t + \ord_p(\alpha p^{N-t}+s) \\
&= \ord_p(k) +  \ord_p(\alpha p^{N-t}+s)
\end{align*}
So it's remaining to show that $\ord_p(\alpha p^{N-t}+s)=0$. If $t<N$, then we have $p\nmid \alpha p^{N-t}+s$. (Otherwise, since $p\mid \alpha p^{N-t}$, so $p\mid s$, which is absurd.) So we are done. And if $t=N$, then $k=p^N\cdot1$, and $\ord_p(\alpha p^{N-t}+s) = \ord_p(\alpha+1)=0$ because $\alpha+1\leq a\leq p-1<p$.

\subsection*{Exercise 1.14}

\subsection*{Exercise 1.15}

\subsection*{Exercise 1.16}

Write $x=p^\alpha a/p^\beta b$ where $p\nmid a,b$. Then $|x|_p=1/p^{\alpha-\beta}\leq1$ iff $\alpha\geq\beta$, i.e., when we write $x$ in the lowest terms, the denominator has no $p$ as its divisor.

\subsection*{Exercise 1.17}

Write $x=p^\alpha a/p^\beta b$ where $p\nmid a,b$, so $\ord_p(x)=\alpha-\beta$. Note that
\begin{align*}
\left|\frac{x^i}{i!}\right|_p &= \prod_{j=1}^i \left|\frac{x}{j}\right|_p = \prod_{j=1}^i p^{-\left(\alpha-\beta-\ord_p(j)\right)} \\
&= p^{-\sum_{j=1}^i \left(\alpha-\beta-\ord_p(j)\right)} = p^{-\left(i(\alpha-\beta)-\ord_p(i!)\right)}
\end{align*}
So the condition $\lim_{i\to\infty} |x^i/i!|_p=0$ is equivalent to $\lim_{i\to\infty} \left(i(\alpha-\beta)-\ord_p(i!)\right)=\infty$. Thus, the problem now becomes showing that
\begin{align*}
\lim_{i\to\infty} \left(i(\alpha-\beta)-\ord_p(i!)\right)=\infty \iff
\begin{cases}
\alpha-\beta\geq 1, &p\neq2 \\
\alpha-\beta\geq 2, &p=2
\end{cases}
\end{align*}
Note also that by Ex 1.14, $\ord_p(i!)=(i-S_i)/(p-1)\geq0$ where $S_i$ is the sum of the digits of $i$ to the base $p$.

Case 1: Assume $p\neq2$. $(\Rightarrow)$ If $\alpha-\beta<1$, then clearly $i(\alpha-\beta)-\ord_p(i!)\not\to\infty$.

$(\Leftarrow)$ Suppose $\alpha-\beta\geq 1$, then
\begin{align*}
i(\alpha-\beta)-\ord_p(i!) &\geq i-\frac{i-S_i}{p-1} = \frac{(p-1)i-i+S_i}{p-1} \\
&\geq \frac{2i-i+S_i}{p-1} = \frac{i+S_i}{p-1} \to \infty
\end{align*}
as $i\to\infty$.

Case 2: Assume $p=2$. $(\Rightarrow)$ Suppose $i(\alpha-\beta)-\ord_p(i!)\to\infty$ as $i\to\infty$. Take $N\in\NN$ s.t. $i(\alpha-\beta)-\ord_p(i!) > 1$ if $i>N$. Choose $s\in\NN$ large enough s.t. $i:=2^s>N$, then $i(\alpha-\beta)-\ord_p(i!) = i(\alpha-\beta-1)+S_i = 2^s(\alpha-\beta-1) +1 >1$. This implies $\alpha-\beta-1>0$ and so $\alpha-\beta\geq 2$.

$(\Leftarrow)$ Suppose $\alpha-\beta\geq 2$, then $i(\alpha-\beta)-\ord_p(i!) = i(\alpha-\beta-1)+S_i \geq i+S_i \to \infty$ as $i\to\infty$.

\subsection*{Exercise 1.18}

Write $x=\pm p_1^{\alpha_1}\cdots p_k^{\alpha_k}$ where $\alpha_i\in\ZZ\setminus\{0\}$ and let $p$ be a prime. If $p\neq p_i$ for all $i$, then $\ord_p(x)=0$ and so $|x|_p=1$. And if $p=p_i$ for some $i$, then $\ord_p(x)=\alpha_i$ and so $|x|_p=1/p_i^{\alpha_i}$. Hence,
\begin{align*}
\prod_p |x|_p := \prod_{p\text{: prime}} |x|_p \cdot |x|_\infty = \prod_{i=1}^k \frac{1}{p_i^{\alpha_i}} \cdot p_1^{\alpha_1}\cdots p_k^{\alpha_k} = 1
\end{align*}

\subsection*{Exercise 1.19}

Fix a prime $p$ and a sequence $\{a_n\}$ of integers. First, by considering $\{a_n\}$ modulo $p$, we see that there are infinitely many terms in $\{a_n\}$ which are all congruent modulo $p$. Collect these terms and call it $\Acal_1$ (regard it as a subsequence of $\{a_n\}$, not a subset). And let $a_{n_1}$ be the first term in $\Acal_1$.

Inductively, suppose we have obtained $a_{n_1},\ldots,a_{n_k}$ in the $k$-th step, $k\geq1$, and the terms in $\Acal_k$ are all congruent modulo $p^k$. Then by considering $\Acal_k\setminus\{a_{n_k}\}$ (omit the first term) modulo $p^{k+1}$, we see that there are infinitely many terms in $\Acal_k\setminus\{a_{n_k}\}$ which are all congruent modulo $p^{k+1}$. Collect these terms and call it $\Acal_{k+1}$. And let $a_{n_{k+1}}$ be the first term in $\Acal_{k+1}$.

By construction, we obtain a descending chain $\{a_n\}\supset\Acal_1\supset\Acal_2\supset\cdots$ of subsequences of $\{a_n\}$ and the desired subsequence $\{a_{n_i}\}$ of $\{a_n\}$, with the properties that for each $i$, the terms in $\Acal_i$ are all congruent modulo $p^i$ and $a_{n_i}\in\Acal_i$. We claim that $\{a_{n_i}\}$ is Cauchy. Given $\epsilon>0$, choose $N$ large enough s.t. $1/p^N<\epsilon$. If $i>j\geq N$, then $a_{n_i},a_{n_j}\in\Acal_j$, i.e., $a_{n_i}\equiv a_{n_j} \pmod{p^j}$. This implies $|a_{n_i}-a_{n_j}|_p \leq 1/p^j \leq 1/p^N < \epsilon$.

\subsection*{Exercise 1.20}

Write $x=\pm p_1^{\alpha_1}\cdots p_k^{\alpha_k}$ where $\alpha_i\in\ZZ\setminus\{0\}$. For each $i$, we have by assumption that $|x|_{p_i} = 1/p_i^{\alpha_i} \leq 1$. This means $\alpha_i \geq 0$ for all $i$ and hence $x\in\ZZ$.

\subsection*{Bonus 1.1}

We show that $\QQ_p$ is complete. (This was left as an exercise in p. 11.) Let $\{a_j\}_{j=1}^\infty$ be a sequence of equivalence classes which is Cauchy in $\QQ_p$. For each $j$, take a representative Cauchy sequence $\{a_{ji}\}_{i=1}^\infty$ of $a_j$ in $\QQ$, where $|a_{ji}-a_{ji'}|<p^{-j}$ if $i,i'\geq N_j$. WLOG, we assume $\{N_j\}$ is strictly increasing.

We first claim that $\{a_{jN_j}\}_{j=1}^\infty$ is Cauchy in $\QQ$. Given $\epsilon>0$, take $M\in\NN$ s.t. $p^{-M}<\epsilon$. Since $\{a_j\}$ is Cauchy in $\QQ_p$, choose $j_\epsilon\in\NN$ s.t. $|a_j-a_{j'}|_p := \lim_{i\to\infty} |a_{ji}-a_{j'i}|_p < \epsilon$ if $j,j'\geq j_\epsilon$. And take $i_\epsilon\in\NN$ s.t. for $i\geq i_\epsilon$, we have $|a_{ji}-a_{j'i}|_p <\epsilon$.

Let $N := \max\{M,j_\epsilon,i_\epsilon\}$. Then if $j>j'\geq N$, we have
\begin{align*}
|a_{jN_j}-a_{j'N_{j'}}|_p \leq \max\{|a_{jN_j}-a_{j'N_j}|_p,|a_{j'N_j}-a_{j'N_{j'}}|_p\} < \max\{\epsilon,p^{-j'}\} = \epsilon
\end{align*}
The second inequality holds because $N_j \geq j > N$, and $N_j>N_{j'}$ by our assumption.

Since $\{a_{jN_j}\}$ is Cauchy in $\QQ$, we let $a\in\QQ_p$ be the equivalence class in which $\{a_{jN_j}\}\in a$. We claim that $\lim_{j\to\infty} a_j=a$, or equivalently, $$\lim_{j\to\infty} |a_j-a|_p = \lim_{j\to\infty}\lim_{i\to\infty} |a_{ji}-a_{iN_i}|_p = 0$$

Given $\epsilon>0$, take $M'\in\NN$ s.t. $p^{-M'}<\epsilon$. Since $\{a_{jN_j}\}$ is Cauchy in $\QQ$, choose $j_\epsilon'\in\NN$ s.t. for $i,j\geq j_\epsilon'$, we have $|a_{iN_i}-a_{jN_j}|_p<\epsilon$. Let $N':=\max\{M',j_\epsilon'\}$. Then if $j\geq N'$, we have $|a_{ji}-a_{iN_i}|_p \leq \max\{|a_{ji}-a_{jN_j}|_p,|a_{jN_j}-a_{iN_i}|_p\}$ by the non-Archimedean property, so $$\lim_{i\to\infty}|a_{ji}-a_{iN_i}|_p \leq \lim_{i\to\infty}\max\{|a_{ji}-a_{jN_j}|_p,|a_{jN_j}-a_{iN_i}|_p\} < \max\{p^{-j},\epsilon\} = \epsilon$$ This completes the proof.

\section*{\S5 (pp. 19-20)}

\subsection*{Exercise 1.1}
We may assume $a_{-m}\neq0$. Since $1\leq a_m \leq p-1$ and $0\leq a_i \leq p-1$ for each $i\geq-m+1$, so
$$-a = (-a_m+p)p^{-m}+(-a_{-m+1}-1+p)p^{-m+1}+\cdots+(-a_0-1+p)+(-a_1-1+p)p+\cdots$$
Note that we have $1\leq -a_m+p\leq p-1$ and $0\leq -a_i-1+p\leq p-1$.

\subsection*{Exercise 1.2}

We omit most of the detailed calculations.

(i) $4+0\cdot7+1\cdot7^2+5\cdot7^3+\cdots$

(ii) $2+0\cdot5+1\cdot5^2+3\cdot5^3+\cdots$

(iii) $8\cdot11^{-1}+8+9\cdot11+5\cdot11^2+\cdots$

(iv) In order to find the $p$-adic expansion of a rational number $a/b$ where $p\nmid b$, we start by writing $a/b=pq+r$ where $q\in\ZZ_p$ and $r=0,1,2,\ldots,p-1$. Then $r$ will be the first term. And repeat this process with $q$ to find the remaing terms. (Make sure you know why this works!)

As an example, let $a/b=2/3$, then we have
\begin{alignat*}{2}
\frac{2}{3}&=2\left(\frac{1}{3}\right)+0 &\qquad\qquad \frac{1}{3}&=2\left(\frac{-1}{3}\right)+1 \\
\frac{-1}{3}&=2\left(\frac{-2}{3}\right)+1 &\qquad\qquad
\frac{-2}{3}&=2\left(\frac{-1}{3}\right)+0
\end{alignat*}
Note that $-1/3$ has already been done in the third step, so we have \begin{align*}
\frac{2}{3} &= 0+1\cdot2+1\cdot2^2+0\cdot2^3+1\cdot2^4+0\cdot2^5+\cdots \\
&= 1\cdot2+1\cdot2^2+1\cdot2^4+1\cdot2^6+\cdots
\end{align*}

(v) Note that the denominator $6=7-1$, and we know $1+p+p^2+\cdots=1/(1-p)$ is valid in the $p$-adic world (see Ex 1.16 if necessary). So $-1/6=1+7+7^2+7^3+\cdots$.

(vi) Similarly, $-1/10=1+11+11^2+\cdots$. So by Ex 1.1, $1/10=10+9\cdot11+9\cdot11^2+\cdots$.

(vii) $-9/16=10+4\cdot13^2+7\cdot13^3+10\cdot13^4+4\cdot13^6+7\cdot13^7+\cdots$

(viii) Note that $1/1000=1/(5^3\cdot8)$. In this situation we need to consider $1/8$ first, then multiply both sides by $1/5^3$ to obtain $1/1000$. Since $1/8=2+4\cdot5+1\cdot5^2+4\cdot5^3+1\cdot5^4+\cdots$, so $1/1000=2\cdot5^{-3}+4\cdot5^{-2}+1\cdot5^{-1}+4+1\cdot5+4\cdot5^2+1\cdot5^3+\cdots$.

(ix) \begin{alignat*}{3}
6!&=3(240)+0 &\qquad\qquad 240&=3(80)+0 &\qquad\qquad 80&=3(26)+2 \\
26&=3(8)+2 &\qquad \qquad 8&=3(2)+2 &\qquad\qquad 2&=3(0)+2 \\
0&=3(0)+0
\end{alignat*}
So $6!=2\cdot3^2+2\cdot3^3+2\cdot3^4+2\cdot3^5$.

(x) $1/3!=1/(3\cdot2)$ and $1/2=2+1\cdot3+1\cdot3^2+1\cdot3^3+\cdots$, so $$\frac{1}{3!}=2\cdot3^{-1}+1+1\cdot3+1\cdot3^2+\cdots$$

(xi) $1/4!=1/(2^3\cdot3)$ and $1/3=1+1\cdot2+1\cdot2^3+1\cdot2^5+\cdots$, so $$\frac{1}{4!}=1\cdot2^{-3}+1\cdot2^{-2}+1+1\cdot2^2+1\cdot2^4+\cdots$$

(xii) $1/5!=1/(5\cdot24)$ and $1/24=4+4\cdot5+3\cdot5^2+4\cdot5^3+3\cdot5^4+\cdots$, so $$\frac{1}{5!}=4\cdot5^{-1}+4+3\cdot5+4\cdot5^2+3\cdot5^3+\cdots$$

\subsection*{Exercise 1.3}

$(\Rightarrow)$ Write $a=\sum_{i=-m}^N a_ip^i$ where $0\leq a_i\leq p-1$ for all $i$. Then $a=\left(\sum_{i=-m}^N a_ip^{i+m}\right)/p^m\in\QQ^+$.

$(\Leftarrow)$ Let $a=b/p^k$ where $b\in\NN$. Write $b$ to the base $p$, $b=\sum_{i=0}^n b_ip^i$ where $0\leq b_i\leq p-1$ for all $i$. Then $a=\sum_{i=0}^n b_ip^{i-k}$.

\subsection*{Exercise 1.4}

$(\Rightarrow)$ Suppose $|a|_p=1/p^m$, $m\in\ZZ$. Write $$ap^m=\jk+\sum_{t=0}^\infty \left(\sum_{s=1}^r a_{N+s}p^{N+s+tr}\right)$$ where $\jk=a_0+a_1p+\cdots+a_Np^N\in\NN$ is the non-repeating part. So we have $$a = \frac{\jk}{p^m}+\frac{\sum_{s=1}^r a_{N+s}p^{N+s}}{(1-p^r)p^m}\in\QQ$$

$(\Leftarrow)$ Write $a=a'/b'$ where $\gcd(a',b')=1$. In view of Ex 1.1, we may assume $a',b'>0$. We can also assume $p\nmid b'$ because the multiplications of $p$ powers preserve the "repeating digits property". Now, let $q_{i-1}=pq_i+r_i$ where $q_0:=a'/b',q_i\in\ZZ_p$ and $r_i=0,1,\ldots,p-1$. It's sufficient to show that $q_i=q_j$ for some $i\neq j$.  (See Ex 1.2 (iv) above.)

We first claim by induction on $i$ that the denominator of $q_i$ is always $b'$ if we write $q_i$ in the lowest terms. When $i=1$, we have $a'/b'=pq_1+r_1$, so $q_1=(a'-r_1b')/pb'$. We know $\alpha_1:=(a'-r_1b')/p\in\ZZ$ because $q_1\in\ZZ_p$. And since $\gcd(a',b')=1$, so is $\gcd(\alpha_1,b')$. So $q_1=\alpha_1/b'$ is in lowest terms.

Suppose this property holds for $i=k-1$. Write $q_{k-1}=\alpha_{k-1}/b'$ where $\gcd(\alpha_{k-1},b')=1$. Consider $q_k=(q_{k-1}-r_k)/p=(\alpha_{k-1}-r_kb')/pb'$. Again, $\alpha_k:=(\alpha_{k-1}-r_kb')/p\in\ZZ$ because $q_k\in\ZZ_p$. And since $\gcd(\alpha_{k-1},b')=1$, so is $\gcd(\alpha_k,b')$. This completes the claim.

By our claim, we can write $q_i=\alpha_i/b'$ in the lowest terms. We next claim by induction on $i$ that $-b'\leq \alpha_i\leq a'$ for all $i$. This will imply that $q_i=q_j$ for some $i\neq j$ by the pigeonhole principle. We will need the fact that $-(p-1)b'\leq-r_ib'\leq 0$ for all $i$, which is easy to see because we have $0\leq r_i\leq p-1$ for all $i$.

When $i=1$, $\alpha_1=b'\cdot q_1=b'\cdot(a'-r_1b')/pb'=(a'-r_1b')/p$. So we have $$-b'\leq\frac{0-(p-1)b'}{p}\leq\alpha_1 = \frac{a'-r_1b'}{p} \leq \frac{a'}{p} \leq a'$$

Suppose this property holds for $i=k-1$, i.e., $-b'\leq\alpha_{k-1}\leq a'$. Note that $\alpha_k=b'\cdot q_k=b'\cdot(q_{k-1}-r_k)/p=b'\cdot(\alpha_{k-1}/b'-r_k)/p=(\alpha_{k-1}-r_kb')/p$. By combining the inequality of $\alpha_{k-1}$ with $-(p-1)b'\leq -r_kb'\leq 0$, we obtain $$-b'=\frac{-b'-(p-1)b'}{p} \leq \alpha_k=\frac{\alpha_{k-1}-r_kb'}{p} \leq \frac{a'+0}{p} \leq a'$$ This completes the proof.

\subsection*{Exercise 1.5}

\subsection*{Exercise 1.6}

We imitate the proof of basic version of Hensel's lemma (p. 16), i.e., we show that there exists $\{a_n\}_{n=1}^\infty\subset\ZZ$ s.t. for all $n$, we have (1) $F(a_n)\equiv0 \pmod{p^{2M+1+n}}$ (2) $a_n\equiv a_{n-1} \pmod{p^{M+n}}$ and (3) $0\leq a_n < p^{M+1+n}$. (See \nameref{Bonus 1.2} for the analog proof of Newton's method.)

Write $a_0 := \tilde{a_0}+\cdots+\tilde{a_M}p^M + \tilde{a_{M+1}}p^{M+1}+\cdots$ where $0\leq\tilde{a_i}\leq p-1$ for all $i$. Set $\tilde{a}:=\tilde{a_0}+\cdots+\tilde{a_M}p^M$, and note that we have $a_0\equiv\tilde{a} \pmod{p^i}$ for $i=1,\ldots,M+1$. Write $F(x)=\sum c_ix^i$. We apply induction on $n$. When $n=1$, (2) and (3) imply $a_1=\tilde{a}+b_1p^{M+1}$ where $0\leq b_1 \leq p-1$. And (1) implies
\begin{align*}
0 &\equiv F(a_1) = \sum c_i\left(\tilde{a}+b_1p^{M+1}\right)^i = \sum c_i\left(\tilde{a}^i+i\tilde{a}^{i-1}b_1p^{M+1}+\jk\right) \\
&\equiv \sum c_i\left(\tilde{a}^i+i\tilde{a}^{i-1}b_1p^{M+1}\right) = F(\tilde{a}) + F'(\tilde{a})b_1p^{M+1} \pmod{p^{2M+2}}
\end{align*}

From $a_0 \equiv \tilde{a} \pmod{p^M}$ we have $F'(\tilde{a})\equiv F'(a_0) \equiv 0 \pmod{p^M}$ by assumption. This implies
\begin{align*}
0\equiv F(a_0) &= \sum c_i\left(\tilde{a} + p^{M+1}\cdot\jk\right)^i \equiv F(\tilde{a}) + F'(\tilde{a})p^{M+1}\cdot\jk \equiv F(\tilde{a}) \pmod{p^{2M+1}}
\end{align*}
So we can write $F(\tilde{a})\equiv \alpha p^{2M+1} \pmod{p^{2M+2}}$ where $\alpha\in\{0,1,\ldots,p-1\}$. Then we have $$\alpha p^{2M+1} + F'(\tilde{a})b_1p^{M+1} \equiv 0 \pmod{p^{2M+2}},$$ or equivalently, $\alpha p^M+F'(\tilde{a})b_1\equiv 0\pmod{p^{M+1}}$. Moreover, since $F'(\tilde{a})\equiv 0 \pmod{p^M}$, so we can further reduce it to $$\alpha+(F'(\tilde{a})/p^M)\cdot b_1 \equiv 0\pmod{p}$$ Finally, since $F'(a_0)\not\equiv 0 \pmod{p^{M+1}}$ by assumption, so $F'(\tilde{a})/p^M \equiv F'(a_0)/p^M\not\equiv 0 \pmod{p}$. Hence, $b_1$ is uniquely solvable.

Now, suppose we have obtained $a_1,\ldots,a_{k-1}$. We want to find $a_k$. (2) and (3) imply $a_k=a_{k-1}+b_kp^{M+k}$ where $0\leq b_k\leq p-1$. And (1) implies
\begin{align*}
0\equiv F(a_k) = \sum c_i\left(a_{k-1}+b_kp^{M+k}\right)^i \equiv F(a_{k-1})+F'(a_{k-1})b_kp^{M+k} \pmod{p^{2M+1+k}}
\end{align*}

Since by (1) in the induction hypothesis, we have $F(a_{k-1})\equiv 0 \pmod{p^{2M+k}}$. So we can write $F(a_{k-1}) \equiv \alpha'p^{2M+k} \pmod{p^{2M+1+k}}$ where
$\alpha'\in\{0,1,\ldots,p-1\}$. Then we have $$\alpha'p^{2M+k}+F'(a_{k-1})b_kp^{M+k} \equiv 0\pmod{p^{2M+1+k}},$$ or equivalently, $\alpha'p^M+F'(a_{k-1})b_k \equiv0 \pmod{p^{M+1}}$. Moreover, from (2) in the induction hypothesis, we have $a_{k-1}\equiv a_0 \pmod{p^{M+1}}$ and so $F'(a_{k-1})\equiv F'(a_0)\equiv 0 \pmod{p^M}$. Hence, we can further reduce it to $$\alpha'+(F'(a_{k-1})/p^M)\cdot b_k \equiv 0\pmod{p}$$ Finally, since $F'(a_{k-1})\equiv F'(a_0)\not\equiv 0 \pmod{p^{M+1}}$, so $F'(a_{k-1})/p^M \equiv F'(a_0)/p^M\not\equiv 0 \pmod{p}$. Hence, $b_k$ is uniquely solvable.

Inductively, we obtain the desired sequence $\{a_n\}$ and $0\leq b_1,b_2,\ldots \leq p-1$ satisfying $a_1=\tilde{a}+b_1p^{M+1}$ and $a_n=a_{n-1}+b_np^{M+n}$ for all $n\geq2$. Now, set $a:=\tilde{a}+b_1p^{M+1}+b_2p^{M+2}+\cdots$. Then $a\equiv \tilde{a} \equiv a_0\pmod{p^{M+1}}$. Moreover, we have for each $n$ that
\begin{align*}
F(a) &\equiv F(\tilde{a}+b_1p^{M+1}+\cdots+b_np^{M+n}) \\
&= F(\tilde{a}+(a_1-\tilde{a})+\cdots+(a_n-a_{n-1})) \\
&= F(a_n) \equiv 0 \pmod{p^{M+1+n}}
\end{align*}
This forces the $p$-adic number $F(a)=0$.

It's remaining to show the uniqueness. This is easy because we know $a$ is of the form $\tilde{a}+b'_1p^{M+1}\cdots$. And every such $a$ corresponds to a sequence $\{a_n\}$ which satisfies (2) and (3). So the uniqueness of $\{a_n\}$ we've just proved implies the uniqness of $a$.

\subsection*{Exercise 1.7}

We can use the process proved in Ex 1.6 to find roots of $F(x)\in\ZZ_p[x]$ in $\QQ_p$. We first pick a nice $M$ (to save our time) and an $a_0$ which satisfies the given three conditions. Next, put $\tilde{a}\equiv a_0 \pmod{p^{M+1}}$ where $0\leq \tilde{a} \leq p^{M+1}-1$. Set $a_1:=\tilde{a}+b_1p^{M+1}$ and $a_n:=a_{n-1}+b_np^{M+n}$. Then find $0\leq b_1,b_2,\ldots \leq p-1$ s.t. $F(a_n)\equiv 0\pmod{p^{2M+1+n}}$. Now, set $a:=\tilde{a}+b_1p^{M+1}+b_2p^{M+2}+\cdots$.

As an example, we find a sqaure root of $-7$ in $\QQ_2$, i.e., a root of $F(x):=x^2+7$. Pick $M=1$ and consider
\begin{align*}
\begin{cases*}
2x\equiv 0 \pmod{2} \\
2x\not\equiv 0 \pmod{2^2} \\
x^2+7\equiv 0 \pmod{2^3}
\end{cases*}
\end{align*}
Choose $a_0=1$ and $\tilde{a}=1\equiv a_0 \pmod{2^{1+1}}$. So $a=1+b_1\cdot2^2+b_2\cdot2^3+b_3\cdot2^4+\cdots$.

When $n=1$, $a_1=1+4b_1$. Consider $$F(a_1)=(1+4b_1)^2+7\equiv 0\pmod{2^4}$$ So $b_1=1$ and $a_1=5$.

When $n=2$, $a_2=5+8b_2$. Consider $$F(a_2)=(5+8b_2)^2+7\equiv 0\pmod{2^5}$$ So $b_2=0$ and $a_2=5$.

When $n=3$, $a_3=5+16b_3$. Consider $$F(a_3)=(5+16b_3)^2+7\equiv 0\pmod{2^6}$$ So $b_3=1$.

Hence, $a=1+1\cdot2^2+1\cdot2^4+\cdots$, or $a=(\cdots10101)_2$, is a square root of $-7$ in $\QQ_2$.

\subsection*{Exercise 1.8}

We use the criterion developed in Ex 1.11: Write $\alpha=a_{-m}p^{-m}+a_{-m+1}p^{-m+1}+\cdots$ as the $p$-adic expansion. If $m$ is odd, then $\alpha$ cannot be a square. Otherwise, we have $\alpha$ is a square in $\QQ_p$ iff $(a_{-m}/p)=1$.

Since the squares in $(\ZZ/11\ZZ)^\times$ are $1,4,9,5,3$, so we have the following: (i) $\bigcirc$ (ii) $\bigtimes$ (iii) $\bigcirc$ (iv) $\bigcirc$ (v) $\bigcirc$ (vi) $\bigtimes$ (vii) $\bigtimes$ (viii) $\bigtimes$ (ix) $\bigcirc$.

\subsection*{Exercise 1.9}

We use the basic version of Hensel's lemma (p. 16).

For $\pm\sqrt{-1}$ in $\QQ_5$. Consider $F(x):=x^2+1\in\ZZ_5[x]$. A simple calculation shows that $a_0=2,3$. For $a_0=2$, we find that
\begin{align*}
F(a_1)&=(2+5b_1)^2+1\equiv 0\pmod{5^2} \implies b_1=1 \implies a_1=7 \\
F(a_2)&=(7+5^2b_2)^2+1\equiv 0\pmod{5^3} \implies b_2=2 \implies a_2=57 \\
F(a_3)&=(57+5^3b_3)^2+1\equiv 0\pmod{5^4} \implies b_3=1
\end{align*}
And for $a_0=3$, we find that
\begin{align*}
F(a_1)&=(3+5b_1)^2+1\equiv 0\pmod{5^2} \implies b_1=3 \implies a_1=18 \\
F(a_2)&=(18+5^2b_2)^2+1\equiv 0\pmod{5^3} \implies b_2=2 \implies a_2=68 \\
F(a_3)&=(68+5^3b_3)^2+1\equiv 0\pmod{5^4} \implies b_3=3
\end{align*}
So $\pm\sqrt{-1}=2+1\cdot5+2\cdot5^2+1\cdot5^3+\cdots$ and $3+3\cdot5+2\cdot5^2+3\cdot5^3+\cdots$ in $\QQ_5$.

For $\pm\sqrt{-3}$ in $\QQ_7$. Consider $G(x):=x^2+3\in\ZZ_7[x]$. A simple calculation shows that $a_0=2,5$. For $a_0=2$, we find that
\begin{align*}
G(a_1)&=(2+7b_1)^2+3\equiv 0\pmod{7^2} \implies b_1=5 \implies a_1=37 \\
G(a_2)&=(37+7^2b_2)^2+3\equiv 0\pmod{7^3} \implies b_2=0 \implies a_2=37\\
G(a_3)&=(37+7^3b_3)^2+3\equiv 0\pmod{7^4} \implies b_3=6
\end{align*}
And for $a_0=5$, we find that
\begin{align*}
G(a_1)&=(5+7b_1)^2+3\equiv 0\pmod{7^2} \implies b_1=1 \implies a_1=12 \\
G(a_2)&=(12+7^2b_2)^2+3\equiv 0\pmod{7^3} \implies b_2=6 \implies a_2=306 \\
G(a_3)&=(306+7^3b_3)^2+3\equiv 0\pmod{7^4} \implies b_3=0
\end{align*}
So $\pm\sqrt{-3}=2+5\cdot7+0\cdot7^2+6\cdot7^3+\cdots$ and $5+1\cdot7+6\cdot7^2+0\cdot7^3+\cdots$ in $\QQ_7$.

\subsection*{Exercise 1.10}

From $1+p+p^2+\cdots=1/(1-p)$ we have $-1=(p-1)+(p-1)p+(p-1)p^2+\cdots$. We split $p$ into two cases.

Case 1: $p$ is an odd prime. Then by the criterion stated in Ex 1.8 (proved in Ex 1.11), it's sufficient to find $p$ s.t. $((p-1)/p)=(-1/p)=1$, or equivalently, $p\equiv1\pmod{4}$. So we find that $p=5,13,17$.

Case 2: $p=2$. Then by the criterion stated (and proved) in Ex 1.12, we see that $-1$ is not a square in $\QQ_2$.

\subsection*{Exercise 1.11}

Let $\alpha\in\ZZ_p^\times$. We claim that $\alpha$ is a square in $\ZZ_p$ iff $\alpha$ is a square modulo $p$. Suppose $\alpha=\beta^2$ where $\beta\in\ZZ_p$. Then $1=|\beta|_p^2$. This implies $\beta\in\ZZ_p^\times$ and $\alpha\equiv \beta^2 \pmod{p}$. Conversely, suppose $\alpha\equiv a^2 \pmod{p}$. Consider $F(x):=x^2-\alpha\in\ZZ_p[x]$. Then $F(a)=a^2-\alpha\equiv0\pmod{p}$ and $F'(a)=2a\not\equiv0 \pmod{p}$ because $a\not\equiv0\pmod{p}$ and $p\neq2$. So by Hensel's lemma (p. 16), $\exists x_0\in\ZZ_p$ s.t. $F(x_0)=0$, i.e., $\alpha=x_0^2$.

If $|\alpha|_p\neq1$, write $\alpha=a_{-m}p^{-m}+a_{-m+1}p^{-m+1}+\cdots$ where $a_{-m}\neq0$ and $m\in\ZZ\setminus\{0\}$. First observe that if $\alpha=\beta^2$ is a square, then $-m=\ord_p(\alpha)=2\ord_p(\beta)$. So if $\ord_p(\alpha)=-m$ is odd, then $\alpha$ is not a square.

Suppose we know $\ord_p(\alpha)$ is even. Consider $p^m\alpha = a_{-m}+a_{-m+1}p+\cdots\in\ZZ_p^\times$. Now, apply the criterion for numbers in $\ZZ_p^\times$, we have that $p^m\alpha$ is a square in $\ZZ_p$ iff $a_{-m}$ is a square modulo $p$, i.e., $(a_{-m}/p)=1$. On the other hand, observe that $p^m\alpha$ is a square in $\ZZ_p$ is also equivalent to say $\alpha$ is a square in $\QQ_p$.

In conclusion, the criterion is as follows: Write $\alpha=a_{-m}p^{-m}+a_{-m+1}p^{-m+1}+\cdots$. If $m$ is odd, then $\alpha$ cannot be a square. Otherwise, we have $\alpha$ is a square in $\QQ_p$ iff $(a_{-m}/p)=1$.

Now, let $0\neq\alpha\in\QQ_p$ be arbitrary. By considering the parities of $m$ and the values of $(a_{-m}/p)$, we see that there are four possibilities of $\alpha$. With this in mind, we take $1\leq b\leq p-1$ s.t. $(b/p)=-1$. And define these four numbers as $1,b,p,bp$. Then we have
\begin{alignat*}{2}
1\cdot\alpha &= a_{-m}p^{-m}+\cdots, &\qquad b\cdot\alpha &= ba_{-m}p^{-m}+\cdots \\
p\cdot\alpha &= a_{-m}p^{-m+1}+\cdots, &\qquad bp\cdot\alpha &= ba_{-m}p^{-m+1}+\cdots
\end{alignat*}
And the following table indicates which number is the only square in the given case:
$$
\def\arraystretch{1.2}
\begin{array}{c|c|c}
 & (a_{-m}/p)=1 & (a_{-m}/p)=-1 \\
\hline
m \text{ is even} & 1\cdot\alpha & b\cdot\alpha \\
\hline
m \text{ is odd} & p\cdot\alpha & bp\cdot\alpha
\end{array}
$$
The idea is that we manipulate $\alpha$ to make its order even, and its first digit square mod $p$.

\subsection*{Exercise 1.12}

Let $\alpha\in\ZZ_2^\times$. We claim that $\alpha$ is a square in $\ZZ_2$ iff
$\alpha\equiv 1\pmod{2^3}$. Suppose $\alpha=\beta^2$ where $\beta\in\ZZ_2$. Then $1=|\beta|_2^2$ and so $\beta\in\ZZ_2^\times$. Write $\beta=1+b_1\cdot2+b_2\cdot2^2+\cdots$ where $b_i\in\{0,1\}$. From this it's easy to check that $\alpha=\beta^2\equiv 1\pmod{2^3}$. Conversely, suppose $\alpha\equiv1\pmod{2^3}$. Consider $F(x):=x^2-\alpha\in\ZZ_2[x]$. Take $M=1$ in Ex 1.6, this means we need to solve
$$
\begin{cases*}
2x \equiv0\pmod{2} \\
2x \not\equiv0\pmod{2^2} \\
x^2-\alpha \equiv x^2-1 \equiv0 \pmod{2^3} \\
\end{cases*}
$$
Clearly $a=1$ works. So by Ex 1.6, $\exists x_0\in\ZZ_2$ s.t. $F(x_0)=0$, i.e., $\alpha=x_0^2$.

If $|\alpha|_2\neq 1$, then the analysis is similar to Ex 1.11. If $\ord_2(\alpha)$ is odd, then $\alpha$ is not a square. Suppose we know $\ord_2(\alpha)$ is even. Write $\alpha=1\cdot2^{-m}+a_{-m+1}\cdot2^{-m+1}+\cdots$ where $a_i\in\{0,1\}$. Consider $2^m\alpha =1+a_{-m+1}\cdot2+\cdots \in\ZZ_2^\times$. Now, apply the criterion for numbers in $\ZZ_2^\times$, we have that $2^m\alpha$ is a square in $\ZZ_2$ iff $2^m\alpha\equiv 1 \pmod{2^3}$, or equivalently, $a_{-m+1}=a_{-m+2}=0$. On the other hand, observe that $2^m\alpha$ is a square in $\ZZ_2$ is also equivalent to say $\alpha$ is a square in $\QQ_2$.

In conclusion, the criterion is as follows: Write $\alpha=1\cdot2^{-m}+a_{-m+1}\cdot2^{-m+1}+\cdots$. If $m$ is odd, then $\alpha$ cannot be a square. Otherwise, we have $\alpha$ is a square in $\QQ_2$ iff $a_{-m+1}=a_{-m+2}=0$.

Now, let $0\neq\alpha\in\QQ_2$ be arbitrary. By considering the parities of $m$ and the values of $a_{-m+1},a_{-m+2}$, we see that there are eight possibilities of $\alpha$. With this in mind, we define these eight numbers as $1,1+2,1+2^2,1+2+2^2,2,(1+2)2,(1+2^2)2,(1+2+2^2)2$. And the following tables indicate which number is the only square in the given case:
$$
\def\arraystretch{1.2}
\begin{array}{c|c|c}
m \text{ is even} & a_{-m+1}=0 & a_{-m+1}=1 \\
\hline
a_{-m+2}=0 & 1\cdot\alpha & (1+2)\cdot\alpha \\
\hline
a_{-m+2}=1 & (1+2^2)\cdot\alpha & (1+2+2^2)\cdot\alpha
\end{array}
\quad
\begin{array}{c|c|c}
m \text{ is odd} & a_{-m+1}=0 & a_{-m+1}=1 \\
\hline
a_{-m+2}=0 & 2\cdot\alpha & (1+2)2\cdot\alpha \\
\hline
a_{-m+2}=1 & (1+2^2)2\cdot\alpha & (1+2+2^2)2\cdot\alpha
\end{array}
$$
The idea is that we manipulate $\alpha$ to make its order even, and the second and the third digits zero.

\subsection*{Exercise 1.13}

Recall that the set of fourth roots of one forms a cyclic group generated by the complex numbers $\pm\sqrt{-1}$. So in order to find their $5$-adic expansions, we first find the $5$-adic expansions of $\pm\sqrt{-1}$, and generate the other three. More precisely, the $5$-adic expansions of $\pm\sqrt{-1}$ has already been found in Ex 1.9. $\pm\sqrt{-1}=2+1\cdot5+2\cdot5^2+1\cdot5^3+\cdots$ and $3+3\cdot5+2\cdot5^2+3\cdot5^3+\cdots$. We pick the former one to be the generator, denoted by $x$. Then the 4 fourth roots of one are
\begin{alignat*}{2}
x&=2+1\cdot5+2\cdot5^2+1\cdot5^3+\cdots &\qquad
x^2&=4+4\cdot5+4\cdot5^2+4\cdot5^3+\cdots = -1\\
x^3&=3+3\cdot5+2\cdot5^2+3\cdot5^3+\cdots &\qquad
x^4&=1+0\cdot5+0\cdot5^2+0\cdot5^3+\cdots = 1
\end{alignat*}

Let $F(x):=x^p-x\in\ZZ_p[x]$. We see that the numbers $0,1,2,\ldots,p-1\in\ZZ_p$ all satisfy $F(x)\equiv0\pmod{p}$ by Fermat's little theorem and $F'(x)\equiv-1\not\equiv0\pmod{p}$. So by Hensel's lemma (p. 16), $\exu a_i\in\ZZ_p$ s.t. $F(a_i)=0$ and $a_i\equiv i\pmod{p}$, $i=0,1,\ldots,p-1$.

\subsection*{Exercise 1.14}

\subsection*{Exercise 1.15}

Assume $p>2$. It's sufficient to show that $f(x):=(x^p-1)/(x-1)$ is irreducible in $\QQ_p[x]$. To apply Eisenstein's criterion, we consider $$g(x):=f(x+1)=\frac{(x+1)^p-1}{x}=x^{p-1}+\binom{p}{1}x^{p-2}+\cdots+\binom{p}{p-2}x+\binom{p}{p-1}$$

Note that we have (1) $\binom{p}{i}\equiv 0\pmod{p}$ for $i=1,2,\ldots,p-1$ (2) $1\not\equiv 0\pmod{p}$ and (3) $\binom{p}{p-1}=p\not\equiv0\pmod{p^2}$. So by Eisenstein's criterion (Ex 1.14), $g(x)$ is irreducible in $\QQ_p[x]$. Consequently, so is $f(x)$. (Otherwise, if $f(x)=f_1(x)f_2(x)$ is reducible, where $1\leq \deg f_1(x),\deg f_2(x) < \deg f(x)$, then so is $g(x)=f(x+1)=f_1(x+1)f_2(x+1)$, which is absurd.)

\subsection*{Exercise 1.16}

Recall that a $p$-adic series converges in $\QQ_p$ iff its terms approach zero (see p. 14). Let $a_n:=p^n$, then $|a_n|_p=1/p^n\to0$ as $n\to\infty$. So $\sum a_n=1+p+p^2+\cdots$ converges to $1/(1-p)$.

Let $b_n:=(-1)^np^n$, then $|b_n|_p=1/p^n\to0$ as $n\to\infty$. So $\sum b_n=1-p+p^2-p^3+\cdots$ converges to $1/(1-(-p))=1/(1+p)$.

Let $c_n:=p^{2n}+(p-1)p^{2n+1}=p^{2n}(p^2-p+1)$, then $|c_n|_p=1/p^{2n}\to0$ as $n\to\infty$. So $\sum c_n=1+(p-1)p+p^2+(p-1)p^3+p^4+(p-1)p^5+\cdots$ converges to $(1+(p-1)p)/(1-p^2)$.

\subsection*{Exercise 1.17}

\subsection*{Exercise 1.18}

For positive $n$, we consider $F(x):=x^n-\alpha\in\ZZ_p[x]$ and show that $F(x)$ has a root in $\ZZ_p$. Put $a_0=1$, then $F(a_0)=1-\alpha\equiv 0\pmod{p}$ and $F'(a_0)=n\not\equiv 0\pmod{p}$. So by Hensel's lemma (p. 16), $\exists a\in\ZZ_p$ s.t. $F(a)=0$.

For negative $n$, we consider $F(x):=\alpha x^{-n}-1\in\ZZ_p[x]$ and show that $F(x)$ has a root in $\ZZ_p$. Put $a_0=1$, then $F(a_0)=\alpha-1\equiv 0\pmod{p}$ and $F'(a_0)=-n\alpha\not\equiv 0\pmod{p}$. So by Hensel's lemma again, $\exists a\in\ZZ_p$ s.t. $F(a)=0$.

Now, suppose $n=p$. Set $\alpha=1+p\equiv1\pmod{p}$. We claim that $x^p-\alpha$ has no solutions in $\QQ_p$, or equivalently, $f(x):=x^p-\alpha$ is irreducible in $\QQ_p[x]$. Using the same idea as in Ex 1.15, we consider $$g(x):=f(x+1)=(x+1)^p-\alpha = x^p+\binom{p}{1}x^{p-1}+\cdots+\binom{p}{p-1}x+1-\alpha$$

Note that we have (1) $\binom{p}{i}\equiv 0\pmod{p}$ for $i=1,2,\ldots,p-1$ (2) $1-\alpha\equiv 0\pmod{p}$ (3) $1\not\equiv 0\pmod{p}$ and (4) $1-\alpha=-p\not\equiv0\pmod{p^2}$. So by Eisenstein's criterion (Ex 1.14), $g(x)$ is irreducible in $\QQ_p[x]$. And hence, so is $f(x)$.

Finally, suppose $\alpha\equiv1\pmod{p^2}$ and $p\neq2$. We claim that $F(x):=x^p-\alpha$ has a solution in $\QQ_p$. Take $M=1$ in Ex 1.6, this means we need to solve
$$
\begin{cases*}
px^{p-1} \equiv0\pmod{p} \\
px^{p-1} \not\equiv0\pmod{p^2} \\
x^p-\alpha \equiv0 \pmod{p^3} \\
\end{cases*}
$$
Write $\alpha=1+0\cdot p+a_2p^2+\cdots$. Then it's easy to check that $a_0:=1+a_2p\in\ZZ_p$ works. (One may observe that $\alpha\equiv 1+a_2p^2\pmod{p^3}$.) So By Ex 1.6, $\exists a\in\ZZ_p$ s.t. $F(a)=0$.

\subsection*{Exercise 1.19}

Induction on $M$. When $M=1$, we have $\alpha^p\equiv\alpha\pmod{p}$ by Fermat's little theorem. Suppose now $\alpha^{p^M}\equiv\alpha^{p^{M-1}} \pmod{p^M}$, we want to show that $\alpha^{p^{M+1}}\equiv\alpha^{p^M} \pmod{p^{M+1}}$. Consider
\begin{align*}
\alpha^{p^{M+1}}-\alpha^{p^M} &= \left(\alpha^{p^M}\right)^p - \left(\alpha^{p^{M-1}}\right)^p \\
&= \left(\alpha^{p^M}-\alpha^{p^{M-1}}\right)\cdot \left(\sum_{i=0}^{p-1} \left(\alpha^{p^M}\right)^{p-1-i}\left(\alpha^{p^{M-1}}\right)^i\right)
\end{align*}
We know $p^M\mid\alpha^{p^M}-\alpha^{p^{M-1}}$ by the induction hypothesis. So it's sufficient to show that $p$ divides that sum. Observe that we have $\alpha\equiv\alpha^p\equiv\alpha^{p^2}\equiv\cdots\equiv\alpha^{p^M} \pmod{p}$. So
\begin{align*}
\sum_{i=0}^{p-1} \left(\alpha^{p^M}\right)^{p-1-i}\left(\alpha^{p^{M-1}}\right)^i \equiv \sum_{i=0}^{p-1} \alpha^{p-1-i}\cdot\alpha^i = \sum_{i=0}^{p-1} \alpha^{p-1} = p\alpha^{p-1} \equiv 0 \pmod{p}
\end{align*}

We next show that $\{a_M\}_{M=1}^\infty:=\{\alpha^{p^M}\}$ is Cauchy in $\QQ_p$. Given $\epsilon>0$, take $N$ large enough s.t. $1/p^N<\epsilon$. If $n>m\geq N$, we have
\begin{align*}
\left|\alpha^{p^n}-\alpha^{p^m}\right|_p &\leq \max\left\{ \left|\alpha^{p^n}-\alpha^{p^{n-1}}\right|_p,\ldots,\left|\alpha^{p^{m+1}}-\alpha^{p^m}\right|_p \right\} \\
&\leq \max \left\{ \frac{1}{p^n},\ldots,\frac{1}{p^{m+1}} \right\} = \frac{1}{p^{m+1}} < \frac{1}{p^N} <\epsilon
\end{align*}
Since $\QQ_p$ is complete, so the sequence $\{a_M\}$ converges to a limit, say $a$, in $\QQ_p$.

Finally, we show that $a$ satisfies $a^p=a$ and $a\equiv\alpha\pmod{p}$. (See Ex 1.13.) From $a_{M-1}^p = (\alpha^{p^{M-1}})^p= \alpha^{p^M} \equiv \alpha^{p^{M-1}} = a_{M-1} \pmod{p^M}$, we have $|a_{M-1}^p-a_{M-1}|_p\leq 1/p^M$. Let $M\to\infty$, we obtain $|a^p-a|_p\leq 0$ and so $a^p=a$. Moreover, from $a_M = \alpha^{p^M} \equiv\alpha\pmod{p}$ for each $M$, we have $|a_M-\alpha|_p\leq 1/p$. Let $M\to\infty$, we obtain $|a-\alpha|_p\leq 1/p$ and so $a\equiv\alpha\pmod{p}$.

\subsection*{Exercise 1.20}

This can be done by using the same idea as in Ex 1.19 in \S2 (p. 8), we skip the details.

\subsection*{Exercise 1.21}

\subsection*{Bonus 1.2} \label{Bonus 1.2}

We give another proof of Ex 1.6 (p. 19), which is an analog of Newton's method.

The given three conditions imply $\ord_p(F'(a_0))=M$ and $\ord_p(F(a_0))>2M$. This means $\ord_p(F(a_0)/F'(a_0))>M$ and so $F(a_0)/F'(a_0)\in\ZZ_p$. Put $a_1:=a_0-F(a_0)/F'(a_0)\in\ZZ_p$. We claim that $\ord_p(F'(a_1))=M$ and $\ord_p(F(a_1))>\ord_p(F(a_0))$.

From $\ord_p(F(a_0)/F'(a_0))>M$ we know $a_1\equiv a_0 \pmod{p^{M+1}}$. So $F'(a_1)\equiv F'(a_0)\equiv0\pmod{p^M}$ and $F'(a_1)\equiv F'(a_0)\not\equiv0\pmod{p^{M+1}}$. This implies $\ord_p(F'(a_1))=M$. Moreover, assume $e_0:=\ord_p(F(a_0))>2M$ and $F(x)=\sum c_ix^i$. Then $\ord_p(F(a_0)/F'(a_0))=e_0-M$ and
\begin{align*}
F(a_1) &= \sum c_i\left(a_0-\frac{F(a_0)}{F'(a_0)}\right)^i = \sum c_i\left(a_0^i - ia_0^{i-1}\frac{F(a_0)}{F'(a_0)}+\jk\right) \\
&\equiv \sum c_i\left(a_0^i - ia_0^{i-1}\frac{F(a_0)}{F'(a_0)}\right) \\
&= F(a_0) - F'(a_0)\cdot\frac{F(a_0)}{F'(a_0)} = 0 \pmod{p^{2(e_0-M)}}
\end{align*}
This implies $\ord_p(F(a_1)) \geq 2(e_0-M) > e_0 = \ord_p(F(a_0)) > 2M$. Note that the second inequality holds because $e_0>2M$.

Inductively, suppose we have obtained $a_k\in\ZZ_p$ satisfies (1) $a_k\equiv a_{k-1} \pmod{p^{M+1}}$, (2) $\ord_p(F'(a_k))=M$ and (3) $\ord_p(F(a_k))>\ord_p(F(a_{k-1}))>2M$. Put $a_{k+1}:=a_k-F(a_k)/F'(a_k)$. One can show that $a_{k+1}\in\ZZ_p$, $a_{k+1}\equiv a_k \pmod{p^{M+1}}$, $\ord_p(F'(a_{k+1}))=M$ and $\ord_p(F(a_{k+1})) > \ord_p(F(a_k))$. (The argument is almost identical to the previous one, we skip the details.)

Hence, we obtain a sequence $\{a_n\}_{n=0}^\infty\subset\ZZ_p$ s.t. for each $n$, $a_n\equiv a_0 \pmod{p^{M+1}}$, $\ord_p(F'(a_n))=M$, and $e_n:=\ord_p(F(a_n))$ is strictly increasing.
This means $\lim_{n\to\infty} e_n=\infty$ and so $\lim_{n\to\infty} |F(a_n)|_p=0$. We claim that $\{a_n\}$ is Cauchy in $\QQ_p$. Given $\epsilon>0$, take $N$ large enough s.t. $p^{M-e_N}<\epsilon$. If $m>n\geq N$, then
\begin{align*}
|a_m-a_n|_p &\leq \max\{|a_m-a_{m-1}|_p,\ldots,|a_{n+1}-a_n|_p\} \\
&= \max\left\{ \left|-\frac{F(a_{m-1})}{F'(a_{m-1})}\right|_p ,\ldots, \left|-\frac{F(a_n)}{F'(a_n)}\right|_p \right\} \\
&= \max\{ p^{M-e_{m-1}},\ldots,p^{M-e_n} \} = p^{M-e_n} \leq p^{M-e_N} < \epsilon
\end{align*}

Since $\QQ_p$ is complete, we set $a:=\lim_{n\to\infty}a_n\in\ZZ_p$. Then $F(a)=F\left(\lim_{n\to\infty} a_n\right) = \lim_{n\to\infty} F(a_n) = 0$. Moreover, from $a_n\equiv a_0 \pmod{p^{M+1}}$ for all $n$, it's obvious that $a\equiv a_0 \pmod{p^{M+1}}$.

It's remaining to show that $a$ is unique. Suppose $\exists b\neq a$ in $\ZZ_p$ also satisfies $F(b)=0$ and $b\equiv a_0\equiv a\pmod{p^{M+1}}$. Write $b=a+c$ where $0\neq c\in\ZZ_p$, then
\begin{align*}
0 &= F(b) = \sum c_i(a+c)^i = \sum c_i(a^i + ia^{i-1}c + c^2 \cdot\jk) \\
&= F(a) + F'(a)c + g(a,c)c^2 = F'(a)c + g(a,c)c^2
\end{align*}
where $g(x,y)\in\ZZ_p[x,y]$. So $F'(a)=-g(a,c)\cdot c$ and hence $$\frac{1}{p^M} = \left|F'(a)\right|_p = |-g(a,c)\cdot c|_p \leq |c|_p = |b-a|_p \leq \frac{1}{p^{M+1}}$$ This leads to a contradiction. (The first equality comes from $\ord_p(F'(a_n))=M$ for all $n$.)

\end{document}