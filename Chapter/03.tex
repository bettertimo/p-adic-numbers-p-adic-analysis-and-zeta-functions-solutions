\documentclass[../Koblitz.tex]{subfiles}

\begin{document}

\chapter{Building up \texorpdfstring{$\Omega$}{Omega}}

\section*{\S2 Extension of norms}

\subsection*{Exercise 3.5}

We first check that $p\ZZ_p$ is a prime ideal in $\ZZ_p$. If $x,y\in\ZZ_p$ and $xy\in p\ZZ_p$, i.e., $|xy|_p=|x|_p|y|_p<1$, then $|x|_p<1$ or $|y|_p<1$. So we have $x\in p\ZZ_p$ or $y\in p\ZZ_p$.

Next, let $P$ be any prime ideal in $\ZZ_p$. We claim that $P=p\ZZ_p$. 

$(\subseteq)$ Let $x\in P\varsubsetneq\ZZ_p$. Since $x$ is not a unit, we have $|x|_p<1$, i.e., $x\in p\ZZ_p$. (Note that we can replace $P$ by any non-zero proper ideal $I$ in $\ZZ_p$ and see that in fact $p\ZZ_p$ is the unique maximal ideal in $\ZZ_p$. This can also be seen by letting $K=\QQ_p$ in the Proposition in page 64.)

$(\supseteq)$ It's sufficient to show that $p\in P$. Take an element $x=p^nu\in P$ where $n\in\NN$ and $u\in\ZZ_p^\times$. Suppose on the contrary, $p\not\in P$, then we have $p^k\not\in P$ for any $k\in\NN$. In particular, $p^n\not\in P$. Moreover, since $u\not\in P$, we have $p^nu\not\in P$, a contradiction.

Lastly, we show that any ideal $I\varsupsetneq \{0\}$ in $\ZZ_p$ is of the form $p^n\ZZ_p$ for some $n\in\NN$. First note that if $0\neq x\in I\subset p\ZZ_p$, then $x=p^mv$ where $m\in\NN$ and $v\in\ZZ_p^\times$. So $p^m=xv^{-1}\in I$. Now, set $n:=\min \{n\mid p^n\in I\}$. (This is possible because the set is non-empty as $\{0\}\varsubsetneq I$.) We claim that $I=p^n\ZZ_p$.

$(\subseteq)$ Let $x=p^mv\in I$ where $m\in\NN$ and $v\in\ZZ_p^\times$. Then by the definition of $n$, we have $m\geq n$. So $x=p^mv=p^n(p^{m-n}v)\in p^n\ZZ_p$.

$(\supseteq)$ It's sufficient to show that $p^n\in I$. Let $x'=p^nw\in I$ having the minimum order $n$ where $w\in\ZZ_p^\times$. Then $p^n=x'\cdot w^{-1}\in I$.

\end{document}